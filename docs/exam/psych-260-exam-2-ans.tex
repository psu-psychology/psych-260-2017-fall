\documentclass[answers]{exam}
\usepackage{graphicx}
\usepackage{wrapfig}
\usepackage[utf8]{inputenc}

\title{PSYCH 260 Exam 2}
\author{}
\date{October 13, 2017}

\pagestyle{headandfoot}
\firstpageheader{PSY 260}{Section 002}{Exam 2}
\runningheader{PSY 260}{Section 002}{Exam 2}
\firstpagefooter{}{Page \thepage\ of \numpages}{}
\runningfooter{}{Page \thepage\ of \numpages}{}

\begin{document}
\maketitle

\begin{center}
  \fbox{\fbox{\parbox{5.5in}{\centering
        Answer the questions using the Scantron form.}}}
\end{center}
\vspace{0.1in}
\makebox[\textwidth]{Name:\enspace\hrulefill}

\newpage

\section{Main}

\textbf{Please put in their proper order the steps that lead to synaptic communication between neurons. Begin with the \emph{presynaptic} cell.}

\begin{questions}

%1
\question Step 1
\begin{choices}
\choice Voltage-gated Ca++ channels open.
\choice Ca++ flow initiates exocytosis of neurotransmitter.
\correctchoice Action potential propagates down the axon to the axon terminal.
\choice Ligand-gated receptors bind neurotransmitter and activate channels in the postsynaptic cell.
\choice Neurotransmitter diffuses across the synaptic cleft.
\end{choices}

%2
\question Step 2
\begin{choices}
\correctchoice Voltage-gated Ca++ channels open.
\choice Ca++ flow initiates exocytosis of neurotransmitter.
\choice Action potential propagates down the axon to the axon terminal.
\choice Ligand-gated receptors bind neurotransmitter and activate channels in the postsynaptic cell.
\choice Neurotransmitter diffuses across synaptic cleft.
\end{choices}

%3
\question Step 3
\begin{choices}
\choice Voltage-gated Ca++ channels open.
\correctchoice Ca++ flow initiates exocytosis of neurotransmitter.
\choice Action potential propagates down the axon to the axon terminal.
\choice Ligand-gated receptors bind neurotransmitter and activate channels in the postsynaptic cell.
\choice Neurotransmitter diffuses across synaptic cleft.
\end{choices}

%4
\question Step 4
\begin{choices}
\choice Voltage-gated Ca++ channels open.
\choice Ca++ flow initiates exocytosis of neurotransmitter.
\choice Action potential propagates down the axon to the axon terminal.
\choice Ligand-gated receptors bind neurotransmitter and activate channels in the postsynaptic cell.
\correctchoice Neurotransmitter diffuses across synaptic cleft.
\end{choices}

%5
\question Step 5
\begin{choices}
\choice Voltage-gated Ca++ channels open.
\choice Ca++ flow initiates exocytosis of neurotransmitter.
\choice Action potential propagates down the axon to the axon terminal.
\correctchoice Ligand-gated receptors bind neurotransmitter and activate channels in the postsynaptic cell.
\choice Neurotransmitter diffuses across synaptic cleft.
\end{choices}

\newpage

\textbf{Answer the following questions.}

% 6
% \question Bergman's rule about the relationship between body mass and latitude suggests that animals get \fillin as average temperatures get colder.
% \begin{choices}
% \choice longer
% \choice smaller.
% \correctchoice larger.
% \choice faster.
% \end{choices}

% \question True or False. There are many obvious structural differences between female and male brains.
% \begin{choices}
% \choice True.
% \correctchoice False.
% \end{choices}

% \question The \fillin of the \fillin control the nervous system's responses to changes in temperature.
% \begin{choices}
% \choice medial geniculate nucleus; thalamus
% \choice inferior colliculus; tegmentum
% \choice postganglionic area; spinal cord
% \correctchoice preoptic area and lateral regions; hypothalamus
% \end{choices}

%7
% \question Influx of \fillin ion into the presynaptic terminal triggers the release of neurotransmitter by means of \fillin.
% \begin{choices}
% \choice $Na^+$; ion transportation.
% \choice Cl-; inhibitory postsynaptic potential enhancement.
% \correctchoice Ca++; exocytosis.
% \choice $K^+$; $Na^+$/$K^+$ pump activity.
% \end{choices}

% \question One theory suggests that the evolution of \fillin helped drive the emergence of complex animal forms during the Cambrian explosion.
% \begin{choices}
% \correctchoice Visual perception.
% \choice Physiological mechanisms for passive feeding.
% \choice Radially symmetric bodies and nervous systems.
% \choice Plants.
% \end{choices}

% \question All of the following "nodes" are part of the Swanson/Cajal Four Systems framework \emph{except}.
% \begin{choices}
% \choice Sensory
% \choice Motor
% \choice Cognition
% \correctchoice Reproduction
% \end{choices}

% 7
\question If a neurotransmitter causes a postsynaptic terminal to \emph{open} a $K^+$ channel, thus increasing the outward flow of this ion, the result will be an \fillin.
\begin{choices}
\choice excitatory pre-synaptic potential
\correctchoice inhibitory post-synaptic potential
\choice enhanced peri-synaptic potential
\choice intrinsic pre-synaptic potential
\end{choices}

%8
% \question All of the following are components of the SAM axis, \emph{except}:
% \begin{choices}
% \correctchoice Midbrain.
% \choice Sympathetic nervous system.
% \choice Adrenal medulla.
% \choice Hypothalamus.
% \end{choices}

% 8 
\question The brain's response to threatening or challenging situations involves both a/an \fillin component (via the release of corticosteroids) and a/an \fillin component (via the sympathetic nervous system).
\begin{choices}
\correctchoice endocrine; neural
\choice dopamine; serotonin
\choice glutamate; GABA
\choice monoamine; amino acid
\end{choices}

\vspace{.75cm}
\textbf{Match the hormone to its function.}

%9
\question Oxytocin
\begin{choices}
\choice stress response; increases blood glucose levels; anti-inflammatory effect.
\correctchoice uterine contraction; milk release; bonding.
\choice regulates seasonal changes; sexual maturation.
\choice blood vessel constriction; antidiuretic hormone.
\end{choices}

%10
\question Cortisol
\begin{choices}
\correctchoice stress response; increases blood glucose levels; anti-inflammatory effect.
\choice uterine contraction; milk release; bonding.
\choice regulates seasonal changes; sexual maturation.
\choice blood vessel constriction; antidiuretic hormone.
\end{choices}

%11
% \question Melatonin
% \begin{choices}
% \choice stress response; increases blood glucose levels; anti-inflammatory effect.
% \choice uterine contraction; milk release; bonding.
% \correctchoice regulates seasonal changes, circadian rhythm; sexual maturation.
% \choice blood vessel constriction; antidiuretic hormone.
% \end{choices}

\newpage

\textbf{Answer the following questions.}

%12
\question Botulinum toxin (botox) blocks the release of acetylcholine (ACh) from presynaptic terminals. In large quantities, this can be \fillin because it \fillin.
\begin{choices}
\choice  good; speeds the conduction of action potentials.
\correctchoice  bad; blocks communication to muscle fibers.
\choice  good; accelerates $K^+$ flow.
\choice  bad; affects the size and number of presynaptic IPSPs.
\end{choices}

%13
\question \fillin is a kind of \fillin brain imaging method used to study axon fiber (white matter) tracts.
\begin{choices}
\choice Structural MRI; structural.
\choice Positron Emission Tomography (PET); functional.
\choice Magnetoencephalography; functional.
\correctchoice diffusion tensor imaging (DTI); structural.
\end{choices}

%14
\question The enzyme AChE contributes to the \fillin of \fillin.
\begin{choices}
\correctchoice  Breakdown and inactivation; acetylcholine.
\choice  Breakdown and inactivation; dopamine, norepinephrine, and epinephrine.
\choice  Postsynaptic reuptake; serotonin.
\choice  Increase in monoamine levels; GABA-releasing neurons.
\end{choices}

%15
\question This neurotransmitter is released by motor neurons onto skeletal muscle.
\begin{choices}
\choice GABA
\choice Serotonin
\correctchoice Acetylcholine
\choice Glutamate
\end{choices}

%16
\question Selective reuptake inhibitors like Prozac act on \fillin, \fillin the normal process of inactivation.
\begin{choices}
\choice  synaptic vesicles; slowing.
\choice  postsynaptic receptors; accelerating.
\correctchoice  presynaptic transporters; slowing.
\choice  $Na^+$/$K^+$ pumps; accelerating.
\end{choices}

%17
\question The meso-limbo-cortical projection from the \fillin in the midbrain releases the neurotransmitter \fillin.
\begin{choices}
\correctchoice ventral tegmental area; dopamine.
\choice raphe nucleus; NE.
\choice superior colliculus; glutamate.
\choice thalamus; GABA.
\end{choices}

\newpage

%18
\question The lateral fissure divides the \fillin.
\begin{choices}
\choice left hemisphere from the right.
\correctchoice temporal lobe from the frontal and parietal lobes.
\choice frontal lobe from the parietal lobes.
\choice corpus callosum from the anterior commissure.
\end{choices}

%19
\question This small cell type contributes to the `pruning' of dendritic spines from unused synapses in the CNS.
\begin{choices}
\choice Pyramidal cells.
\correctchoice microglia.
\choice Schwann cells.
\choice Stellate cells.
\end{choices}

%20
\question \fillin receptors do \emph{not} contain their own ion channel.
\begin{choices}
\choice ionotropic
\correctchoice metabotropic
\choice ligand-gated
\choice voltage-gated
\end{choices}

%21
\question \fillin is the primary \emph{excitatory} neurotransmitter in the CNS; \fillin is the primary neurotransmitter of \emph{CNS output}.
\begin{choices}
\choice GABA; glutamate.
\choice glutamate; GABA.
\correctchoice glutamate; acetylcholine.
\choice Acetylcholine; glutamate.
\end{choices}

%22
\question Hormonal action \fillin than neuronal action.
\begin{choices}
\choice is faster-acting.
\choice is more specific in its effects.
\correctchoice is slower-acting.
\choice involves greater voluntary control.
\end{choices}

% 23
\question Opening a channel permeable to $Na^+$ in a neuron at its resting potential would have a/an \fillin effect.
\begin{choices}
\correctchoice excitatory.
\choice inhibitory.
\choice modulatory.
\choice Ca++ activating.
\end{choices}

\newpage

\textbf{Match the endocrine structure with the function.}

% 24
\question Hypothalamus
\begin{choices}
\choice Circadian rhythms.
\choice Responds to adrenocoricotropic hormone (ACTH) by releasing cortisol.
\choice Releases NE and epinephrine.
\correctchoice Controls hormone secretions into and by pituitary.
\end{choices}

% 25
% \question Pineal gland
% \begin{choices}
% \correctchoice Circadian rhythms.
% \choice Responds to adrenocoricotropic hormone (ACTH) by releasing cortisol.
% \choice Releases NE and epinephrine.
% \choice Controls hormone secretions into and by pituitary.
% \end{choices}

%26
\question Adrenal cortex
\begin{choices}
\choice Influences circadian rhythms by releasing melatonin.
\correctchoice Releases cortisol.
\choice Releases 5-HT, NE, and epinephrine.
\choice Controls hormone secretions into and by pituitary.
\end{choices}

\newpage

\textbf{Match the Roman numeral in the figure below, to the processes and structures in the hormonal action cycle the figure depicts.}

\begin{figure}[h]
\includegraphics[width=0.50\textwidth]{img/oxytocin.jpg}
\centering
\end{figure}

%27
\question I
\begin{choices}
\choice Nerve impulses activate temporal lobe neurons.
\correctchoice Nerve impulses activate the hypothalamus.
\choice Nerve impluses activate the anterior pituitary.
\choice Somatosensory cortex activates the thalamus.
\end{choices}

%28
\question II
\begin{choices}
\choice Posterior pituitary.
\choice Hippocampus.
\choice Anterior pituitary.
\correctchoice Hypothalamus.
\end{choices}

%29
\question III
\begin{choices}
\correctchoice Posterior pituitary.
\choice Anterior pituitary.
\choice Pineal gland.
\choice Hypothalamus.
\end{choices}

%30
\question IV
\begin{choices}
\choice Release of GnRH into blood stream.
\choice Release of melatonin into blood stream.
\correctchoice Release of oxytocin into blood stream.
\choice Release of cortisol into blood stream.
\end{choices}

\newpage

\textbf{Answer the following questions.}

%31
\question Both Parkinson's Disease and schizophrenia have been linked to disturbances in \fillin neurotransmitter systems.
\begin{choices}
\correctchoice  dopamine.
\choice  GABA.
\choice  acetylcholine.
\choice  serotonin.
\end{choices}

%32
% \question All of the following are \emph{biologically} driven ``periods'' of animal physiology \emph{except}.
% \begin{choices}
% \correctchoice once a week.
% \choice 90-110 min.
% \choice daily.
% \choice yearly.
% \end{choices}

%32
\question The human neural tube begins to form at about \fillin weeks of gestation, eventually becoming the \fillin. 
\begin{choices}
\choice 13; peripheral nervous system
\choice 40; autonomic nervous system
\correctchoice 3; cerebral ventricles \& central canal of the spinal cord
\choice 1; cerebral aqueduct of the midbrain
\end{choices}

%33
% \question The \fillin plays a role in entraining the release of the hormone \fillin to patterns in the day/night cycle.
% \begin{choices}
% \choice hippocampus; adrenaline.
% \choice preoptic area; ACTH.
% \correctchoice suprachiasmatic nucleus; melatonin.
% \choice amygdala; glutamate.
% \end{choices}

% 33
\question The release of the circadian-rhythm-regulating hormone \fillin from the \fillin is controlled by a sympathetic nervous system neuron which releases \fillin as a neurotransmitter.
\begin{choices}
\correctchoice melatonin; pineal gland; norepinephrine
\choice melanin; posterior pituitary; GABA
\choice vasopressin; anterior pituitary; dopamine
\choice norepinephrine; adrenal cortex; serotonin
\end{choices}

%34
% \question \fillin sleep is characterized by large amplitude, low frequency EEG patterns and the absence of vivid dream experiences.
% \begin{choices}
% \correctchoice slow-wave (Stage 3/4).
% \choice REM.
% \choice Stage 1.
% \choice Stage 2.
% \end{choices}

% 34
\question A chemical released by one neuron onto another neuron is called a \fillin while one released by a neuron into the bloodstream is called a \fillin.
\begin{choices}
\choice tropic hormone; releasing hormone
\choice reuptake inhibitor; endocrine enhancer
\correctchoice neurotransmitter; hormone
\choice ligand-gated channel; voltage-gated channel
\end{choices}

%35
% \question One reason young infants might have such erratic sleep patterns is that
% \begin{choices}
% \correctchoice it takes time for retinal neurons to establish effective connections with the SCN.
% \choice most new parents don't keep regular day/night schedules.
% \choice they spend most of their in slow-wave sleep.
% \choice the infant hypothalamus doesn't release melatonin like an adult.
% \end{choices}

%35
\question Specialized molecules embedded in the presynaptic membrane called transporters contribute to the \fillin phase of neurotransmitter release.
\begin{choices}
\correctchoice inactivation
\choice action potential
\choice voltage-gated $Ca^++$ exit
\choice second messenger signaling
\end{choices}

%36
% \question During REM sleep, most motor neurons are \fillin -- except for those controlling the \fillin.
% \begin{choices}
% \choice inhibited; limbs.
% \choice excited; eyes.
% \correctchoice inhibited; eyes.
% \choice excited; digestive tract.
% \end{choices}

%36
\question One feature of the human brain that now appears especially distinctive and important in explaining our cognitive capacity is the \fillin.
\begin{choices}
\correctchoice number of neurons in the cerebral cortex
\choice number of neurons in the cerebellum
\choice the size of the cerebellum
\choice the speed of action potential propagation
\end{choices}

%37
% \question In which group of animals is monogamy common?
% \begin{choices}
% \choice primates.
% \correctchoice birds.
% \choice mammals.
% \choice reptiles.
% \end{choices}

% 37
\question The first animals with neurons and nervous systems emerged around the time of the \fillin, about \fillin years ago.
\begin{choices}
\choice "Big Bang"; 13.8 billion
\choice formation of the Earth; 4.6 billion
\correctchoice Cambrian Explosion; 540 million
\choice end of the last Ice Age; 12,000
\end{choices}

\newpage

%38
% \question Human sexuality differs from most other animals in all of the following ways \emph{except}.
% \begin{choices}
% \choice Have sex outside of estrous.
% \choice Fewer outward signs of estrous.
% \correctchoice Smaller testes, ejaculate volumes, and sperm counts.
% \choice Have sex frequently.
% \end{choices}

% 38
% \question Which neurotransmitter/hormone is involved in sexual arousal, childbirth, and social bonding?
% \begin{choices}
% \choice Serotonin.
% \correctchoice Oxytocin.
% \choice Epinephrine.
% \choice Vasopressin.
% \end{choices}

% \question Which of these is NOT one of features of the human brain that contributes to our greater processing capacity?
% \begin{choices}
% \choice Dense interconnections.
% \choice High levels of myelination.
% \correctchoice Larger mass but fewer folds.
% \choice Large cerebral cortex.
% \end{choices}

\question Cortical areas in humans have maximal \emph{synaptic density} \fillin.
\begin{choices}
\choice in the 30s and 40s
\choice in adolescence
\choice prenatally
\correctchoice before the age of 5.
\end{choices}

% 38
\question The formation of synapses (synaptogenesis) \fillin; myelination \fillin.
\begin{choices}
\correctchoice continues long after birth; also continues long after birth.
\choice continues long after birth; stops before birth.
\choice ends before birth; continues long after birth.
\choice ends before birth; also ends before birth.
\end{choices}

% 39
\question Across the animal kingdom, bigger animals generally have \fillin brains.
\begin{choices}
\correctchoice bigger.
\choice smaller.
\choice smooother, less wrinkled.
\choice radially symmetric.
\end{choices}

\question The vast majority of neurons and glia in the CNS are generated \fillin from a set of precursor cells that line the \fillin.
\begin{choices}
\correctchoice prenatally; neural tube
\choice prenatally; synaptic vesicles
\choice postnatally; neural tube
\choice postnatally; synaptic vesicles
\end{choices}

% \question The human \fillin is/are disproportionately large in comparison to other primates.
% \begin{choices}
% \choice cerebellum.
% \choice spinal cord.
% \choice cerebral ventricles.
% \correctchoice cerebral cortex.
% \end{choices}

%39
\question Gap junctions support \fillin between cells.
\begin{choices}
\correctchoice direct electrical coupling
\choice chemical communication
\choice slow communication
\choice hormonal signaling
\end{choices}

%40
\question The release of glutamate onto an AMPA receptor on a neuron's dendrite produces an \fillin.
\begin{choices}
\choice inhibitory postsynaptic potential (IPSP)
\choice electrochemical postsynaptic potential (EPSP)
\choice inwardly-driven postsynaptic potential (IPSP)
\correctchoice excitatory postsynaptic potential (EPSP)
\end{choices}

\vspace{2cm}
\begin{center}
\textbf{Turn to the next page to complete the bonus questions.}
\end{center}

\newpage
\section{Bonus}

%41
\question The hippocampus is located deep within which lobe of the cerebral cortex?
\begin{choices}
\correctchoice Temporal.
\choice Frontal.
\choice Parietal.
\choice Occipital.
\end{choices}

%42
% \question Histamine is one of the \fillin group of neurotransmitters. It is released by the \fillin.
% \begin{choices}
% \choice monoamine; hippocampus.
% \correctchoice monoamine; hypothalamus.
% \choice amino acid; midbrain.
% \choice peptide; amygdala.
% \end{choices}

\question All of these brain development processes show patterns of increase and decline in the first several months (or years) of life \emph{except}.
\begin{choices}
\correctchoice myelination
\choice synaptogenesis
\choice thickness of cerebral cortex
\choice number of neurons in spinal cord.
\end{choices}

%43
\question The 10th cranial (Xth) or vagus nerve connects to the \fillin branch of the autonomic nervous system. Its neurons tend to slow heart rate and activate digestion when stimulated.
\begin{choices}
\correctchoice parasympathetic.
\choice sympathetic.
\choice enteric.
\choice somatic.
\end{choices}

%44
\question Corticotropin Releasing Hormone (CRH) is released by the \fillin into the \fillin.
\begin{choices}
\choice hippocampus; amygdala.
\choice adrenal cortex; blood stream.
\correctchoice hypothalamus; anterior pituitary.
\choice medulla oblongata; adrenal medulla.
\end{choices}

\end{questions}
\end{document}
